\section{Introduzione}
SmartVend è un software di gestione di una rete di vending machines smart che mira a rendere il processo di utilizzo, manutenzione e approvvigionamento dei prodotti più semplice ed efficiente possibile. Il software si occuperà di mantenere aggiornati i log riguardanti le vendite e i problemi riscontrati sui singoli distributori in modo da poter identificare problematiche comuni da risolvere su determinati modelli di distributori e tenere traccia dei prodotti che vengono più apprezzati in determinate zone geografiche.

\subsection{Statement}
L'obiettivo principale del sistema è fornire una soluzione per la gestione efficiente delle vending machines, con funzionalità specifiche per diverse tipologie di utenti: Clienti (Customer), Manutentori (Worker) e Amministratori (Admin).

L'applicazione consentirà all'utente di tipo \textbf{Customer} di acquistare prodotti connettendosi al distributore con il proprio dispositivo e di ricaricare il proprio borsellino elettronico pagando con carta o contanti.\\

L'utente di tipo \textbf{Worker} (Manutentore) avrà la possibilità di visualizzare l'elenco degli interventi a lui assegnati con la relativa descrizione del compito da svolgere e di notificare il completamento di una riparazione o approvvigionamento di un distributore. Potrà inoltre segnalare problemi o guasti.\\

L'utente di tipo \textbf{Admin} (Amministratore) avrà responsabilità più ampie, tra cui la gestione degli utenti (CRUD), inclusa l'aggiunta, la rimozione e l'aggiornamento dei dettagli dei manutentori. L'amministratore potrà anche gestire i distributori automatici (CRUD), consentendo l'aggiunta, la rimozione e l'aggiornamento delle loro informazioni, e gestire gli articoli disponibili nei distributori (CRUD), permettendo di aggiungere, rimuovere e aggiornare gli articoli. Infine, avrà accesso a statistiche sulle vendite e sulle performance delle macchine.\\

L'obiettivo primario è offrire un'esperienza utente intuitiva e funzionale, garantendo al contempo un'elevata manutenibilità e la possibilità di future espansioni. Il sistema è progettato per ottimizzare la gestione operativa dei distributori automatici, riducendo i tempi di inattività e migliorando l'efficienza complessiva del servizio.